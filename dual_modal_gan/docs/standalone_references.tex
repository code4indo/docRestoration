% ============================================================
% DOKUMEN STANDALONE DAFTAR PUSTAKA
% ============================================================
% File ini digunakan untuk menghasilkan PDF terpisah hanya untuk daftar pustaka

\documentclass[12pt,a4paper]{article}

% Package setup
\usepackage[utf8]{inputenc}
\usepackage[T1]{fontenc}
\usepackage[bahasa]{babel}
\usepackage{mathptmx} % Times New Roman font
\usepackage{microtype} % Untuk optimasi typography dan spacing

% Menonaktifkan hyphenation secara total untuk Bahasa Indonesia
\lefthyphenmin=62
\righthyphenmin=62
\pretolerance=10000
\tolerance=2000
\emergencystretch=10pt
\hbadness=10000
\vbadness=10000
\raggedright
\hyphenpenalty=10000
\exhyphenpenalty=10000
\linepenalty=10000
\binoppenalty=10000
\relpenalty=10000
\usepackage{geometry}
\usepackage{hyperref}

% Geometry setup sesuai format tesis ITB
\geometry{
    left=4cm,    % Tepi dalam (sesuai ketentuan tesis)
    right=3cm,   % Tepi luar
    top=3cm,     % Tepi atas
    bottom=3cm   % Tepi bawah
}

% Line spacing 1.5
\renewcommand{\baselinestretch}{1.5}

% Hyperref setup
\hypersetup{
    pdfencoding=auto,
    pdftitle={Daftar Pustaka},
    pdfauthor={Tesis},
    pdfsubject={Daftar Pustaka},
    colorlinks=true,
    linkcolor=black,
    citecolor=black,
    urlcolor=blue,
    pdfstartview=FitH
}

% Document info
\title{\textbf{DAFTAR PUSTAKA}}
\author{}
\date{}

\begin{document}

% Title
\begin{center}
    \Large\textbf{DAFTAR PUSTAKA}
\end{center}

\vspace{1cm}

% Import references dari file terpisah
% ============================================================
% DAFTAR PUSTAKA
% ============================================================

% File ini berisi daftar pustaka yang dapat digunakan oleh semua bab
% Format: APA 7th Edition - Bahasa Indonesia
% Diimpor menggunakan % ============================================================
% DAFTAR PUSTAKA
% ============================================================

% File ini berisi daftar pustaka yang dapat digunakan oleh semua bab
% Format: APA 7th Edition - Bahasa Indonesia
% Diimpor menggunakan % ============================================================
% DAFTAR PUSTAKA
% ============================================================

% File ini berisi daftar pustaka yang dapat digunakan oleh semua bab
% Format: APA 7th Edition - Bahasa Indonesia
% Diimpor menggunakan \input{references.tex} di akhir dokumen

\begin{thebibliography}{99}
\setlength{\itemsep}{0.2cm}

% --- Referensi Utama GAN dan Document Enhancement ---

\bibitem{goodfellow2014}
Goodfellow, I., dkk. (2014).
\textit{Generative Adversarial Nets}.
Advances in Neural Information Processing Systems, 27, 2672-2680.

\bibitem{erb2019}
Souibgui, M., dkk. (2019).
\textit{Enhance to Read Better: A Multi-Task Adversarial Network for Handwritten Document Image Enhancement}.
Pattern Recognition Letters, 128, 115-122.

\bibitem{degan2021}
Souibgui, M. A., \& Kessentini, Y. (2021).
\textit{DE-GAN: A Conditional Generative Adversarial Network for Document Enhancement}.
IEEE Transactions on Emerging Topics in Computational Intelligence, 3(1), 13-25.

\bibitem{docentr2022}
Souibgui, M., dkk. (2022).
\textit{DocEnTr: An End-to-End Document Image Enhancement Transformer}.
IEEE Conference on Computer Vision and Pattern Recognition Workshops, 322-327.

\bibitem{textdiae2022}
Souibgui, M., dkk. (2022).
\textit{Text-DIAE: A Self-Supervised Degradation Invariant Autoencoder for Text Recognition and Document Enhancement}.
Proceedings of the AAAI Conference on Artificial Intelligence, 36(3), 3026-3034.

% --- Referensi Multi-modal Learning ---

\bibitem{baltrusaitis2019}
Baltrusaitis, T., Ahuja, C., \& Morency, L. P. (2019).
\textit{Multimodal Machine Learning: A Survey and Taxonomy}.
IEEE Transactions on Pattern Analysis and Machine Intelligence, 41(2), 423-443.

% --- Referensi Handwritten Text Recognition (HTR) ---

\bibitem{graves2006}
Graves, A., dkk. (2006).
\textit{Connectionist Temporal Classification: Labelling Unsegmented Sequence Data with Recurrent Neural Networks}.
Proceedings of the 23rd International Conference on Machine Learning, 369-376.

% --- Referensi Document Binarization dan Traditional Methods ---

\bibitem{otsu1979}
Otsu, N. (1979).
\textit{A Threshold Selection Method from Gray-Level Histograms}.
IEEE Transactions on Systems, Man, and Cybernetics, 9(1), 62-66.

\bibitem{sauvola2000}
Sauvola, J., \& Pietikainen, M. (2000).
\textit{Adaptive Document Image Binarization}.
Pattern Recognition, 33(2), 225-236.

\bibitem{gatos2006}
Gatos, B., dkk. (2006).
\textit{Adaptive Degraded Document Image Binarization}.
Pattern Recognition, 39(3), 317-327.

% --- Referensi Document Quality and HTR Performance ---

\bibitem{jadhav2022}
Jadhav, A., dkk. (2022).
\textit{Correlation Analysis Between Visual Quality Metrics and HTR Performance in Document Restoration}.
Document Recognition and Retrieval, 13420, 134-143.

% --- Referensi Arsip dan Preservasi Digital ---

\bibitem{unesco2003}
UNESCO. (2003).
\textit{Memory of the World Register: UNESCO's Programme for the Preservation and Access to Documentary Heritage}.
Paris: UNESCO Publishing.

\bibitem{uu2009}
Republik Indonesia. (2009).
\textit{Undang-Undang Nomor 43 Tahun 2009 tentang Kearsipan}.
Lembaran Negara Republik Indonesia Tahun 2009 Nomor 152.

% --- Referensi Transformer Architecture ---

\bibitem{vaswani2017}
Vaswani, A., dkk. (2017).
\textit{Attention is All You Need}.
Advances in Neural Information Processing Systems, 30, 5998-6008.

\end{thebibliography} di akhir dokumen

\begin{thebibliography}{99}
\setlength{\itemsep}{0.2cm}

% --- Referensi Utama GAN dan Document Enhancement ---

\bibitem{goodfellow2014}
Goodfellow, I., dkk. (2014).
\textit{Generative Adversarial Nets}.
Advances in Neural Information Processing Systems, 27, 2672-2680.

\bibitem{erb2019}
Souibgui, M., dkk. (2019).
\textit{Enhance to Read Better: A Multi-Task Adversarial Network for Handwritten Document Image Enhancement}.
Pattern Recognition Letters, 128, 115-122.

\bibitem{degan2021}
Souibgui, M. A., \& Kessentini, Y. (2021).
\textit{DE-GAN: A Conditional Generative Adversarial Network for Document Enhancement}.
IEEE Transactions on Emerging Topics in Computational Intelligence, 3(1), 13-25.

\bibitem{docentr2022}
Souibgui, M., dkk. (2022).
\textit{DocEnTr: An End-to-End Document Image Enhancement Transformer}.
IEEE Conference on Computer Vision and Pattern Recognition Workshops, 322-327.

\bibitem{textdiae2022}
Souibgui, M., dkk. (2022).
\textit{Text-DIAE: A Self-Supervised Degradation Invariant Autoencoder for Text Recognition and Document Enhancement}.
Proceedings of the AAAI Conference on Artificial Intelligence, 36(3), 3026-3034.

% --- Referensi Multi-modal Learning ---

\bibitem{baltrusaitis2019}
Baltrusaitis, T., Ahuja, C., \& Morency, L. P. (2019).
\textit{Multimodal Machine Learning: A Survey and Taxonomy}.
IEEE Transactions on Pattern Analysis and Machine Intelligence, 41(2), 423-443.

% --- Referensi Handwritten Text Recognition (HTR) ---

\bibitem{graves2006}
Graves, A., dkk. (2006).
\textit{Connectionist Temporal Classification: Labelling Unsegmented Sequence Data with Recurrent Neural Networks}.
Proceedings of the 23rd International Conference on Machine Learning, 369-376.

% --- Referensi Document Binarization dan Traditional Methods ---

\bibitem{otsu1979}
Otsu, N. (1979).
\textit{A Threshold Selection Method from Gray-Level Histograms}.
IEEE Transactions on Systems, Man, and Cybernetics, 9(1), 62-66.

\bibitem{sauvola2000}
Sauvola, J., \& Pietikainen, M. (2000).
\textit{Adaptive Document Image Binarization}.
Pattern Recognition, 33(2), 225-236.

\bibitem{gatos2006}
Gatos, B., dkk. (2006).
\textit{Adaptive Degraded Document Image Binarization}.
Pattern Recognition, 39(3), 317-327.

% --- Referensi Document Quality and HTR Performance ---

\bibitem{jadhav2022}
Jadhav, A., dkk. (2022).
\textit{Correlation Analysis Between Visual Quality Metrics and HTR Performance in Document Restoration}.
Document Recognition and Retrieval, 13420, 134-143.

% --- Referensi Arsip dan Preservasi Digital ---

\bibitem{unesco2003}
UNESCO. (2003).
\textit{Memory of the World Register: UNESCO's Programme for the Preservation and Access to Documentary Heritage}.
Paris: UNESCO Publishing.

\bibitem{uu2009}
Republik Indonesia. (2009).
\textit{Undang-Undang Nomor 43 Tahun 2009 tentang Kearsipan}.
Lembaran Negara Republik Indonesia Tahun 2009 Nomor 152.

% --- Referensi Transformer Architecture ---

\bibitem{vaswani2017}
Vaswani, A., dkk. (2017).
\textit{Attention is All You Need}.
Advances in Neural Information Processing Systems, 30, 5998-6008.

\end{thebibliography} di akhir dokumen

\begin{thebibliography}{99}
\setlength{\itemsep}{0.2cm}

% --- Referensi Utama GAN dan Document Enhancement ---

\bibitem{goodfellow2014}
Goodfellow, I., dkk. (2014).
\textit{Generative Adversarial Nets}.
Advances in Neural Information Processing Systems, 27, 2672-2680.

\bibitem{erb2019}
Souibgui, M., dkk. (2019).
\textit{Enhance to Read Better: A Multi-Task Adversarial Network for Handwritten Document Image Enhancement}.
Pattern Recognition Letters, 128, 115-122.

\bibitem{degan2021}
Souibgui, M. A., \& Kessentini, Y. (2021).
\textit{DE-GAN: A Conditional Generative Adversarial Network for Document Enhancement}.
IEEE Transactions on Emerging Topics in Computational Intelligence, 3(1), 13-25.

\bibitem{docentr2022}
Souibgui, M., dkk. (2022).
\textit{DocEnTr: An End-to-End Document Image Enhancement Transformer}.
IEEE Conference on Computer Vision and Pattern Recognition Workshops, 322-327.

\bibitem{textdiae2022}
Souibgui, M., dkk. (2022).
\textit{Text-DIAE: A Self-Supervised Degradation Invariant Autoencoder for Text Recognition and Document Enhancement}.
Proceedings of the AAAI Conference on Artificial Intelligence, 36(3), 3026-3034.

% --- Referensi Multi-modal Learning ---

\bibitem{baltrusaitis2019}
Baltrusaitis, T., Ahuja, C., \& Morency, L. P. (2019).
\textit{Multimodal Machine Learning: A Survey and Taxonomy}.
IEEE Transactions on Pattern Analysis and Machine Intelligence, 41(2), 423-443.

% --- Referensi Handwritten Text Recognition (HTR) ---

\bibitem{graves2006}
Graves, A., dkk. (2006).
\textit{Connectionist Temporal Classification: Labelling Unsegmented Sequence Data with Recurrent Neural Networks}.
Proceedings of the 23rd International Conference on Machine Learning, 369-376.

% --- Referensi Document Binarization dan Traditional Methods ---

\bibitem{otsu1979}
Otsu, N. (1979).
\textit{A Threshold Selection Method from Gray-Level Histograms}.
IEEE Transactions on Systems, Man, and Cybernetics, 9(1), 62-66.

\bibitem{sauvola2000}
Sauvola, J., \& Pietikainen, M. (2000).
\textit{Adaptive Document Image Binarization}.
Pattern Recognition, 33(2), 225-236.

\bibitem{gatos2006}
Gatos, B., dkk. (2006).
\textit{Adaptive Degraded Document Image Binarization}.
Pattern Recognition, 39(3), 317-327.

% --- Referensi Document Quality and HTR Performance ---

\bibitem{jadhav2022}
Jadhav, A., dkk. (2022).
\textit{Correlation Analysis Between Visual Quality Metrics and HTR Performance in Document Restoration}.
Document Recognition and Retrieval, 13420, 134-143.

% --- Referensi Arsip dan Preservasi Digital ---

\bibitem{unesco2003}
UNESCO. (2003).
\textit{Memory of the World Register: UNESCO's Programme for the Preservation and Access to Documentary Heritage}.
Paris: UNESCO Publishing.

\bibitem{uu2009}
Republik Indonesia. (2009).
\textit{Undang-Undang Nomor 43 Tahun 2009 tentang Kearsipan}.
Lembaran Negara Republik Indonesia Tahun 2009 Nomor 152.

% --- Referensi Transformer Architecture ---

\bibitem{vaswani2017}
Vaswani, A., dkk. (2017).
\textit{Attention is All You Need}.
Advances in Neural Information Processing Systems, 30, 5998-6008.

\end{thebibliography}

\end{document}