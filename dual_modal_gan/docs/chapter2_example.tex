% ============================================================
% BAB II: TINJAUAN PUSTAKA (CONTOH FILE TERPISAH)
% ============================================================

\documentclass[12pt,a4paper]{article}
\usepackage[utf8]{inputenc}
\usepackage[T1]{fontenc}
\usepackage[bahasa]{babel}
\usepackage{mathptmx}
\usepackage{microtype}
\usepackage{graphicx}
\usepackage{amsmath}
\usepackage{amssymb}
\usepackage{setspace}
\usepackage{geometry}
\usepackage{tikz}
\usepackage{float}
\usepackage{enumitem}
\usepackage{bookmark}
\usepackage{hyperref}
\usepackage{tocloft}
\usepackage{caption}
\usepackage{textcomp}

% Pengaturan margin sesuai ITB
\geometry{
    a4paper,
    left=4cm,
    right=3cm,
    top=3cm,
    bottom=3cm
}

% Pengaturan spasi
\onehalfspacing

% Pengaturan paragraf
\setlength{\parindent}{0pt}
\setlength{\parskip}{0pt}

% Pengaturan hyperlink
\hypersetup{
    colorlinks=true,
    linkcolor=black,
    citecolor=black,
    urlcolor=blue
}

\begin{document}

\vspace{2cm}
\begin{center}
{\fontsize{14}{16.8}\selectfont\textbf{BAB II. Tinjauan Pustaka}}\\[1em]
\end{center}
\label{sec:tinjauan-pustaka}

\subsection{Generative Adversarial Networks (GAN)}

\textit{Generative Adversarial Networks} (GAN) diperkenalkan oleh Goodfellow et al. (2014) sebagai framework pembelajaran yang terdiri dari dua jaringan neural yang bersaing: generator dan discriminator. Generator mencoba menghasilkan data sintetis yang realistis, sementara discriminator berusaha membedakan data asli dengan data palsu. Proses kompetitif ini memungkinkan GAN untuk mempelajari distribusi data yang kompleks dan menghasilkan output berkualitas tinggi.

\subsubsection{Aplikasi GAN dalam Restorasi Dokumen}

Dalam konteks restorasi dokumen historis, GAN telah menunjukkan performa superior dibandingkan metode tradisional. Souibgui, Khlif, Amara, \& El Abed (2019) memperkenalkan ERB-MultiTaskAdversarial yang merupakan penelitian pionir dalam mengintegrasikan GAN dengan CRNN untuk evaluasi keterbacaan. Namun, metode ini memiliki keterbatasan dalam integrasi end-to-end.

\subsubsection{DE-GAN: Pendekatan Berbasis Conditional GAN}

Souibgui \& Kessentini (2021) mengembangkan DE-GAN dengan pendekatan conditional GAN yang secara khusus dirancang untuk dokumen terdegradasi berat. DE-GAN melaporkan penurunan CER yang signifikan dari 0.37 menjadi 0.01, namun evaluasi keterbacaan dilakukan secara pasca-pelatihan bukan sebagai bagian dari fungsi objektif.

\subsubsection{DocEnTr: Transformer-only Approach}

Souibgui et al. (2022) memperkenalkan DocEnTr yang menggunakan arsitektur transformer-only tanpa komponen adversarial. Meskipun mencapai PSNR tertinggi (35.2 dB), pendekatan ini cenderung menghasilkan output yang over-smoothed dan kehilangan detail tekstur penting untuk HTR.

\subsection{Handwritten Text Recognition (HTR)}

\subsubsection{Connectionist Temporal Classification (CTC)}

Graves et al. (2006) memperkenalkan Connectionist Temporal Classification (CTC) yang memungkinkan training model sequence-to-sequence tanpa alignment manual. CTC loss menjadi standar dalam evaluasi HTR dan penting untuk integrasi dengan framework restorasi dokumen.

\subsection{Multi-modal Learning}

Baltrusaitis, Ahuja, \& Morency (2019) menunjukkan bahwa arsitektur multi-modal dapat meningkatkan stabilitas training dan kualitas output dengan menggabungkan informasi dari berbagai modality. Pendekatan ini relevan untuk pengembangan diskriminator dual-modal yang mengevaluasi koherensi visual-tekstual.

% Import referensi dari file terpisah yang sama dengan Bab 1
% ============================================================
% DAFTAR PUSTAKA
% ============================================================

% File ini berisi daftar pustaka yang dapat digunakan oleh semua bab
% Format: APA 7th Edition - Bahasa Indonesia
% Diimpor menggunakan % ============================================================
% DAFTAR PUSTAKA
% ============================================================

% File ini berisi daftar pustaka yang dapat digunakan oleh semua bab
% Format: APA 7th Edition - Bahasa Indonesia
% Diimpor menggunakan % ============================================================
% DAFTAR PUSTAKA
% ============================================================

% File ini berisi daftar pustaka yang dapat digunakan oleh semua bab
% Format: APA 7th Edition - Bahasa Indonesia
% Diimpor menggunakan \input{references.tex} di akhir dokumen

\begin{thebibliography}{99}
\setlength{\itemsep}{0.2cm}

% --- Referensi Utama GAN dan Document Enhancement ---

\bibitem{goodfellow2014}
Goodfellow, I., dkk. (2014).
\textit{Generative Adversarial Nets}.
Advances in Neural Information Processing Systems, 27, 2672-2680.

\bibitem{erb2019}
Souibgui, M., dkk. (2019).
\textit{Enhance to Read Better: A Multi-Task Adversarial Network for Handwritten Document Image Enhancement}.
Pattern Recognition Letters, 128, 115-122.

\bibitem{degan2021}
Souibgui, M. A., \& Kessentini, Y. (2021).
\textit{DE-GAN: A Conditional Generative Adversarial Network for Document Enhancement}.
IEEE Transactions on Emerging Topics in Computational Intelligence, 3(1), 13-25.

\bibitem{docentr2022}
Souibgui, M., dkk. (2022).
\textit{DocEnTr: An End-to-End Document Image Enhancement Transformer}.
IEEE Conference on Computer Vision and Pattern Recognition Workshops, 322-327.

\bibitem{textdiae2022}
Souibgui, M., dkk. (2022).
\textit{Text-DIAE: A Self-Supervised Degradation Invariant Autoencoder for Text Recognition and Document Enhancement}.
Proceedings of the AAAI Conference on Artificial Intelligence, 36(3), 3026-3034.

% --- Referensi Multi-modal Learning ---

\bibitem{baltrusaitis2019}
Baltrusaitis, T., Ahuja, C., \& Morency, L. P. (2019).
\textit{Multimodal Machine Learning: A Survey and Taxonomy}.
IEEE Transactions on Pattern Analysis and Machine Intelligence, 41(2), 423-443.

% --- Referensi Handwritten Text Recognition (HTR) ---

\bibitem{graves2006}
Graves, A., dkk. (2006).
\textit{Connectionist Temporal Classification: Labelling Unsegmented Sequence Data with Recurrent Neural Networks}.
Proceedings of the 23rd International Conference on Machine Learning, 369-376.

% --- Referensi Document Binarization dan Traditional Methods ---

\bibitem{otsu1979}
Otsu, N. (1979).
\textit{A Threshold Selection Method from Gray-Level Histograms}.
IEEE Transactions on Systems, Man, and Cybernetics, 9(1), 62-66.

\bibitem{sauvola2000}
Sauvola, J., \& Pietikainen, M. (2000).
\textit{Adaptive Document Image Binarization}.
Pattern Recognition, 33(2), 225-236.

\bibitem{gatos2006}
Gatos, B., dkk. (2006).
\textit{Adaptive Degraded Document Image Binarization}.
Pattern Recognition, 39(3), 317-327.

% --- Referensi Document Quality and HTR Performance ---

\bibitem{jadhav2022}
Jadhav, A., dkk. (2022).
\textit{Correlation Analysis Between Visual Quality Metrics and HTR Performance in Document Restoration}.
Document Recognition and Retrieval, 13420, 134-143.

% --- Referensi Arsip dan Preservasi Digital ---

\bibitem{unesco2003}
UNESCO. (2003).
\textit{Memory of the World Register: UNESCO's Programme for the Preservation and Access to Documentary Heritage}.
Paris: UNESCO Publishing.

\bibitem{uu2009}
Republik Indonesia. (2009).
\textit{Undang-Undang Nomor 43 Tahun 2009 tentang Kearsipan}.
Lembaran Negara Republik Indonesia Tahun 2009 Nomor 152.

% --- Referensi Transformer Architecture ---

\bibitem{vaswani2017}
Vaswani, A., dkk. (2017).
\textit{Attention is All You Need}.
Advances in Neural Information Processing Systems, 30, 5998-6008.

\end{thebibliography} di akhir dokumen

\begin{thebibliography}{99}
\setlength{\itemsep}{0.2cm}

% --- Referensi Utama GAN dan Document Enhancement ---

\bibitem{goodfellow2014}
Goodfellow, I., dkk. (2014).
\textit{Generative Adversarial Nets}.
Advances in Neural Information Processing Systems, 27, 2672-2680.

\bibitem{erb2019}
Souibgui, M., dkk. (2019).
\textit{Enhance to Read Better: A Multi-Task Adversarial Network for Handwritten Document Image Enhancement}.
Pattern Recognition Letters, 128, 115-122.

\bibitem{degan2021}
Souibgui, M. A., \& Kessentini, Y. (2021).
\textit{DE-GAN: A Conditional Generative Adversarial Network for Document Enhancement}.
IEEE Transactions on Emerging Topics in Computational Intelligence, 3(1), 13-25.

\bibitem{docentr2022}
Souibgui, M., dkk. (2022).
\textit{DocEnTr: An End-to-End Document Image Enhancement Transformer}.
IEEE Conference on Computer Vision and Pattern Recognition Workshops, 322-327.

\bibitem{textdiae2022}
Souibgui, M., dkk. (2022).
\textit{Text-DIAE: A Self-Supervised Degradation Invariant Autoencoder for Text Recognition and Document Enhancement}.
Proceedings of the AAAI Conference on Artificial Intelligence, 36(3), 3026-3034.

% --- Referensi Multi-modal Learning ---

\bibitem{baltrusaitis2019}
Baltrusaitis, T., Ahuja, C., \& Morency, L. P. (2019).
\textit{Multimodal Machine Learning: A Survey and Taxonomy}.
IEEE Transactions on Pattern Analysis and Machine Intelligence, 41(2), 423-443.

% --- Referensi Handwritten Text Recognition (HTR) ---

\bibitem{graves2006}
Graves, A., dkk. (2006).
\textit{Connectionist Temporal Classification: Labelling Unsegmented Sequence Data with Recurrent Neural Networks}.
Proceedings of the 23rd International Conference on Machine Learning, 369-376.

% --- Referensi Document Binarization dan Traditional Methods ---

\bibitem{otsu1979}
Otsu, N. (1979).
\textit{A Threshold Selection Method from Gray-Level Histograms}.
IEEE Transactions on Systems, Man, and Cybernetics, 9(1), 62-66.

\bibitem{sauvola2000}
Sauvola, J., \& Pietikainen, M. (2000).
\textit{Adaptive Document Image Binarization}.
Pattern Recognition, 33(2), 225-236.

\bibitem{gatos2006}
Gatos, B., dkk. (2006).
\textit{Adaptive Degraded Document Image Binarization}.
Pattern Recognition, 39(3), 317-327.

% --- Referensi Document Quality and HTR Performance ---

\bibitem{jadhav2022}
Jadhav, A., dkk. (2022).
\textit{Correlation Analysis Between Visual Quality Metrics and HTR Performance in Document Restoration}.
Document Recognition and Retrieval, 13420, 134-143.

% --- Referensi Arsip dan Preservasi Digital ---

\bibitem{unesco2003}
UNESCO. (2003).
\textit{Memory of the World Register: UNESCO's Programme for the Preservation and Access to Documentary Heritage}.
Paris: UNESCO Publishing.

\bibitem{uu2009}
Republik Indonesia. (2009).
\textit{Undang-Undang Nomor 43 Tahun 2009 tentang Kearsipan}.
Lembaran Negara Republik Indonesia Tahun 2009 Nomor 152.

% --- Referensi Transformer Architecture ---

\bibitem{vaswani2017}
Vaswani, A., dkk. (2017).
\textit{Attention is All You Need}.
Advances in Neural Information Processing Systems, 30, 5998-6008.

\end{thebibliography} di akhir dokumen

\begin{thebibliography}{99}
\setlength{\itemsep}{0.2cm}

% --- Referensi Utama GAN dan Document Enhancement ---

\bibitem{goodfellow2014}
Goodfellow, I., dkk. (2014).
\textit{Generative Adversarial Nets}.
Advances in Neural Information Processing Systems, 27, 2672-2680.

\bibitem{erb2019}
Souibgui, M., dkk. (2019).
\textit{Enhance to Read Better: A Multi-Task Adversarial Network for Handwritten Document Image Enhancement}.
Pattern Recognition Letters, 128, 115-122.

\bibitem{degan2021}
Souibgui, M. A., \& Kessentini, Y. (2021).
\textit{DE-GAN: A Conditional Generative Adversarial Network for Document Enhancement}.
IEEE Transactions on Emerging Topics in Computational Intelligence, 3(1), 13-25.

\bibitem{docentr2022}
Souibgui, M., dkk. (2022).
\textit{DocEnTr: An End-to-End Document Image Enhancement Transformer}.
IEEE Conference on Computer Vision and Pattern Recognition Workshops, 322-327.

\bibitem{textdiae2022}
Souibgui, M., dkk. (2022).
\textit{Text-DIAE: A Self-Supervised Degradation Invariant Autoencoder for Text Recognition and Document Enhancement}.
Proceedings of the AAAI Conference on Artificial Intelligence, 36(3), 3026-3034.

% --- Referensi Multi-modal Learning ---

\bibitem{baltrusaitis2019}
Baltrusaitis, T., Ahuja, C., \& Morency, L. P. (2019).
\textit{Multimodal Machine Learning: A Survey and Taxonomy}.
IEEE Transactions on Pattern Analysis and Machine Intelligence, 41(2), 423-443.

% --- Referensi Handwritten Text Recognition (HTR) ---

\bibitem{graves2006}
Graves, A., dkk. (2006).
\textit{Connectionist Temporal Classification: Labelling Unsegmented Sequence Data with Recurrent Neural Networks}.
Proceedings of the 23rd International Conference on Machine Learning, 369-376.

% --- Referensi Document Binarization dan Traditional Methods ---

\bibitem{otsu1979}
Otsu, N. (1979).
\textit{A Threshold Selection Method from Gray-Level Histograms}.
IEEE Transactions on Systems, Man, and Cybernetics, 9(1), 62-66.

\bibitem{sauvola2000}
Sauvola, J., \& Pietikainen, M. (2000).
\textit{Adaptive Document Image Binarization}.
Pattern Recognition, 33(2), 225-236.

\bibitem{gatos2006}
Gatos, B., dkk. (2006).
\textit{Adaptive Degraded Document Image Binarization}.
Pattern Recognition, 39(3), 317-327.

% --- Referensi Document Quality and HTR Performance ---

\bibitem{jadhav2022}
Jadhav, A., dkk. (2022).
\textit{Correlation Analysis Between Visual Quality Metrics and HTR Performance in Document Restoration}.
Document Recognition and Retrieval, 13420, 134-143.

% --- Referensi Arsip dan Preservasi Digital ---

\bibitem{unesco2003}
UNESCO. (2003).
\textit{Memory of the World Register: UNESCO's Programme for the Preservation and Access to Documentary Heritage}.
Paris: UNESCO Publishing.

\bibitem{uu2009}
Republik Indonesia. (2009).
\textit{Undang-Undang Nomor 43 Tahun 2009 tentang Kearsipan}.
Lembaran Negara Republik Indonesia Tahun 2009 Nomor 152.

% --- Referensi Transformer Architecture ---

\bibitem{vaswani2017}
Vaswani, A., dkk. (2017).
\textit{Attention is All You Need}.
Advances in Neural Information Processing Systems, 30, 5998-6008.

\end{thebibliography}

\end{document}